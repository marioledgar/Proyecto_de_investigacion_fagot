\PassOptionsToPackage{es-noquoting}{babel}
\documentclass[11pt]{scrartcl}

\usepackage{graphicx}

\usepackage[bib,spanish]{mario}
% \usepackage{appendix} %% Solo si no es KOMA-script
\usepackage{subcaption}
\usepackage{array}
\usepackage{subfiles}
\usepackage[justification=centering]{caption}

\addto\captionsspanish{
\def\tablename{Tabla}
}

\usepackage{biblatex}
\addbibresource{biblio.bib}

\usepackage{csquotes}

\usepackage{siunitx} \sisetup{input-digits = 0123456789\pi} % Para que deje usar \pi en siunitx

\renewcommand{\arraystretch}{1.5}

\title{Análisis de los armónicos del fagot}
\author{Mario Ledesma García}
\date{Abril 2025}

\begin{document}

\maketitle

% % Secciones
% - Portada con el título y los nombres y apellidos.
% - Abstract en la portada, justo debajo del título y los nombres.
% - Índice paginado.
% Hipótesis.
% 1. Introducción teórica.
% 2. Procedimiento experimental.
% 2.1. Materiales.
% 2.2. Proceso paso a paso.
% 3. Presentación de resultados.
% 4. Discusión de resultados.
% 5. Conclusiones.
% 6. Bibliografía.

\section{Introducción} \label{intro} \subfile{partes/intro/intro}

\section{Marco teórico} \label{marco} \subfile{partes/marco_teorico/marco_teorico}

\appendix
\section{Demostraciones} \label{demostraciones} \subfile{partes/apendices/demostraciones}

%https://koppreeds.com/harmonic.html

%https://onlinetonegenerator.com/multiple-tone-generator.html
%Generador de sonidos

\printbibliography

\end{document}
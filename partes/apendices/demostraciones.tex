\documentclass[../main.tex]{subfiles}

\begin{document}

\subsection{Ecuación de la onda armónica}

Para llegar a la ecuación \ref{eq:ecuacion_mas} tenemos que primero encontrar la ecuación del punto $x = 0$.
Como es la ecuación de un MAS, es
\begin{equation*}
  y(0, t) = A \sen(\omega t + \varphi_0)
\end{equation*}
donde $A$, $\omega$ y $\varphi_0$ son, respectivamente, la amplitud, la frecuencia angular y el desfase del MAS en $x = 0$ y, por tanto, la amplitud, la frecuencia angular y el desfase de la onda.

Ahora, introducimos la variable $x$ en la ecuación.
Por ahora vamos a suponer que la onda se propaga en el sentido positivo.
Como sabemos que la velocidad viene dada por
\[v = \frac{\text{distancia}}{\text{tiempo}},\]
tenemos que la onda tarda
\[\frac{x}{v}\]
en recorrer el espacio entre el origen y un punto a distancia $x$.
Por tanto, tenemos que
\[y(0, t) = y\left(x, t + \frac{x}{v}\right).\]
Poniendo esto en la ecuación nos queda
\begin{align*}
  y(x, t) & = y \left(0, t - \frac{x}{v}\right)                                      \\
          & = A \sen \left( \omega \left(t - \frac{x}{v} \right) + \varphi_0 \right) \\
          & = A \sen \left( \omega (t v -  x) + \varphi_0 \right)                    \\
          & = A \sen \left( \omega t - x + \varphi_0 \right)
\end{align*}



\subsection{Convergencia de la serie de Fourier}

En esta sección derivamos la expresión de los coeficientes de la serie de Fourier.
Suponemos que el periodo de $f$ es $2 \pi$.

Empezamos con la siguiente igualdad:
\begin{equation}\label{eq:suma_fourier}
  f(x) = \divdos{a_0} + \sum_{n=1}^\infty \left(a_n \cos(nx) + b_n\sen(nx)\right).
\end{equation}
Nos gustaría conseguir aislar un solo coeficiente, para eso usamos las fórmulas de ortogonalidad del seno y el coseno.

\begin{lemma*}[Fórmulas de ortogonalidad del seno y coseno]
  Para números enteros $n$, $m \geq 0$ se cumple que
  \begin{align*}
    \int_0^{2 \pi} \sen(nx)\sen(mx) \ddint x & =
    \begin{cases}
      0   & n \not= m \\
      \pi & n = m
    \end{cases}                                                             \\
    \int_0^{2\pi} \cos(nx) \cos(mx) \ddint x & =
    \begin{cases}
      0     & n \not= m     \\
      \pi   & n = m \not= 0 \\
      2 \pi & n = m = 0
    \end{cases}                                                       \\
    \int_0^{2\pi} \sen(nx) \cos(mx) \ddint x & = 0 \hspace{3mm} \forall\, n, m.
  \end{align*}
\end{lemma*}

\begin{subproof}
  Recordemos primero las siguientes identidades (fórmulas producto-suma):
  \begin{align*}
    2\sen x \sen y & = \cos(x - y) - \cos(x + y)  \\
    2\cos x \cos y & = \cos(x - y) + \cos(x + y)  \\
    2\sen x \cos y & = \sen(x + y) + \sen(x - y).
  \end{align*}
  Empezamos con la primera de las anteriores igualdades
  \begin{align}
    \int_0^{2 \pi} \sen(nx)\sen(mx) \ddint x & = \half \int_0^{2 \pi} \cos\left( (n - m) x \right) + \cos \left( (n + m) x \right) \ddint x \nonumber \,                       \\
                                             & = \half \left[ \int_0^{2\pi} \cos ((n - m) x) \ddint x + \int_0^{2\pi} \cos((n + m) x) \ddint x \right] \label{eq:integral_fea}
  \end{align}

  Por un momento, supongamos que $n \not= m$.

  Es fácil ver que para cualquier $n \not= 0$ se cumple que
  \[\int_0^{2\pi} \sen (nx) \ddint x = \int_0^{2\pi} \cos (nx) \ddint x = 0.\]
  Esto se puede ver visualmente en la figura \ref{fig:integral_seno}: las áreas naranjas y las verdes se cancelan.
  Lo mismo pasa con el coseno, solo que con cierto desfase.
  \begin{figure}[ht]
    \centering
    \includegraphics[width=0.5\linewidth]{figuras/integral_seno.png}
    \caption{Integral entre $0$ y $2\pi$ de $\sen(2x)$. Elaboración propia.}
    \label{fig:integral_seno}
  \end{figure}

  En el caso $n \not= m$, queda claro que la integral es $0$.
  En el caso $n = m$ nos queda que es $1$.

  Se pueden demostrar las otras dos relaciones de ortogonalidad de manera similar.
\end{subproof}

Ahora, fijamos $k \geq 0$ y multiplicamos ambos lados de la ecuación \ref{eq:suma_fourier} por $\cos(kx)$ e integramos:
\begin{equation*}
  \int_0^{2 \pi} f(x)\cos(kx) \ddint x = a_k \pi
\end{equation*}
\begin{equation*}
  a_n = \frac{1}{\pi} \int_0^{2 \pi}f(x)\cos(nx) \ddint x.
\end{equation*}

Multiplicando por $\sen(kx)$ e integrando nos queda que
\begin{equation*}
  b_n = \frac{1}{\pi} \int_0^{2 \pi} f(x) \sen(nx) \ddint x.
\end{equation*}

Por último, si el periodo no es $2\pi$, definimos una nueva función
\[f_0(x) = f\left(\frac{2 \pi}{P}x\right).\]
Podemos hacer la serie de Fourier de esta nueva función, y es fácil comprobar que al volver a la función inicial, los coeficientes serían
\begin{align*}
  a_n & = \frac{2}{P} \int_{-P/2}^{P/2} f(x)\cos \left( \frac{2 \pi n}{P}x \right) \ddint x, \hspace{1cm} \text{para} \hspace{1.2mm} n = 0, 1, 2, \dots \\
  b_n & = \frac{2}{P} \int_{-P/2}^{P/2} f(x)\sen \left( \frac{2 \pi n}{P}x \right) \ddint x, \hspace{1.03cm} \text{para} \hspace{1.2mm} n = 1, 2, \dots
\end{align*}


\subsection{Equivalencia de la forma compleja de la  serie de Fourier}

Para demostrar que la forma compleja y la trigonométrica de la serie de Fourier son equivalentes, nos basamos en la fórmula de Euler, presentamos sin demostración:
\begin{eq*}
  e^{i x} = \cos(x) + i \sen (x)
\end{eq*}
para cualquier número real $x$.

Suponemos otra vez que $P = 2 \pi$, por comodidad.
Recordemos la definición de los coeficientes complejos de Fourier $c_n$:
\[c_n = \frac{1}{2\pi} \int_{0}^{2\pi} f(x) e^{-inx} \ddint x, \hspace{1cm} \text{para} \hspace{1.2mm} n = \dots, -2, -1, 0, 1, 2, \dots \]
La clave está en que $c_n e^{inx} + c_{-n}e^{-inx} = a_n \cos(nx) + b_n\sen(nx)$:



% \begin{align*}
%     c_ne^{inx} + c_{-n}e^{-inx} &= \frac{1}{2\pi} \int_{0}^{2\pi} f(x) e^{-inx} \ddint x \cdot e^{inx} + \frac{1}{2\pi} \int_{0}^{2\pi} f(x) e^{inx} \ddint x \cdot e^{-inx}\\
%     &= \frac{1}{2\pi} \int_{0}^{2\pi} f(x) e^{-inx} \ddint x \cdot e^{inx} + \frac{1}{2\pi} \int_{0}^{2\pi} f(x) e^{inx} \ddint x \cdot e^{-inx}\\
% \end{align*}


% https://math.stackexchange.com/questions/5084389/extension-of-fourier-series-to-fourier-transform

% atenuación ondas (p. 151 del libro de 2º bach)

\end{document}